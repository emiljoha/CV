% cover letter LaTeX template\
\documentclass[10pt]{letter}
\usepackage{newcent}
\usepackage[swedish]{babel}
\usepackage[T1]{fontenc}
\usepackage[utf8]{inputenc}
\signature{Emil Johansson}
\address{Bredgatan 24C \\ Lund \\ Sverige}
\begin{document}
\begin{letter}{Bildgruppen \\ Matematik LTH \\ Lund\\ Sverige}
%\opening{Dear Sir or Madam:}
  \opening{Hej!}

  Sedan jag först hörde talas om artificiella neuronnät har de
  fascinerat mig. Enkla beståndsdelar som tillsammans bildar en
  förvånansvärt generell icke-linjär funktion. Att inspirationen
  kommer från biologin gör det bara ännu mer spännande.

  Mitt masterarbete inleddes i början av september och fortsätter fram
  till min examen i början av juni. Som del av examensarbete använder
  jag teknologier som Python, TensorFlow, Keras, och Google Compute
  Engine. Målet är att använda neuralnät för optimering med
  bivillkor. Projektet är ett samarbete mellan fasta tillståndets
  fysik, matematisk fysik och teoretisk biofysik. Mina
  programmeringskunskaper är goda och jag är bekväm med GNU/Linux
  baserade operativsystem.

  Genom min utbildning inom naturvetenskaplig fysik med inriktning mot
  teoretisk fysik och matematik har jag skaffat mig en stabil
  matematisk grund. Naturvetenskaplig fysik handlar i grund och botten
  om matematisk modellering, utbildningen har således byggt upp min
  förmåga att analysera och abstrahera modeller.

  Jag har inte en standard teknisk fysik/matematik bakgrund men jag ser
  det som en fördel. Mitt yttre må inte bidra till någon mångfald men
  min utbildningsbakgrund gör.

  Ser fram emot att träffa er och få berätta mer om mig själv.

  \closing{Vänliga Hälsningar,}
%\closing{Yours Faithfully,}
% \ps{PS. Är medveten om att tillträdet är första mars vid vilken tid
%   jag inte är klar med mitt examensarbete.}
%\encl{CV and/or Resume}
\end{letter}
\end{document}

%%% Local Variables:
%%% mode: latex
%%% TeX-master: t
%%% End:
