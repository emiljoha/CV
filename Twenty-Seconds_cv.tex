\documentclass[]{twentysecondcv}
\usepackage{hyperref}
\begin{document}

%%%%%%%%%%%%%%%%%
%%PROFILE SIDE BAR%%
%%%%%%%%%%%%%%%%%

%%%%%%%%%%%%%%%%
%%PERSONAL INFO%%%
%%%%%%%%%%%%%%%%

\profilepic{emil.jpg} %path of profile pic
\cvname{Emil Johansson} %your name
\cvjobtitle{Student at Lund University}%your actual job position
\cvdate{22 August 1994}%date of birth
\cvaddress{Bredgatan 24 C, 22221 Lund, Sweden}%Address
\cvnumberphone{+46 736251509}%telephone number
\cvmail{Emil.Johansson.073@student.lu.se}%e-mail
\cvsite{http://github.com/Nat13Jo}%personal site


\aboutme{
  Mathematical Physicist with a passion for Software Engineering. I hope
  to find a position combining mathematics software engineering.

  Favorite language: c++ |
  Least favorite language: Matlab |
  Favorite editor: Emacs
} 
%%%%%%%%%%%%%%%%%%%%%%%%%%%%%%%%%%%%%%%%%%%%%%%%%%%%%%%%%%%%%%
%%%%%%Skill bar section, each skill must have a value between 0 an 6 (float)%%%%%%%
%%%%%%%%%%%%%%%%%%%%%%%%%%%%%%%%%%%%%%%%%%%%%%%%%%%%%%%%%%%%%%
\skills{{Respect of non-deserving authority/1},{Respect of deserving
    authority/6},{quantum mechanics/5},
  {git/4},{SVN/3},{c++/4},{python/4},{mathematics/5}}

%%%%%%%%%%%%%%%%%%%%%%%%%%%%%%%%%%%%%%%%%%%%%%%%%%%%%%%%%%%%%%
%%%%%%Skill text section, each skill must have a value between 0 an 6%%%%%%%%%%%%
%%%%%%%%%%%%%%%%%%%%%%%%%%%%%%%%%%%%%%%%%%%%%%%%%%%%%%%%%%%%%%
%\skillstext{{lovely/4},{narcissistic/3}}

\makeprofile
%%%%%%%%%%%%%%%%%%%%
%%END PROFILE SIDE BAR%%
%%%%%%%%%%%%%%%%%%%%

%%%%%%%%%%%%%%%%%%%%
%%%%%%%%BODY%%%%%%%%
%%%%%%%%%%%%%%%%%%%%

%%%%%%%%%%%%%%%%%%%%
%%SIMPLE SECTION%%%%%%
%%%%%%%%%%%%%%%%%%%%
\section{interests}
In my spare time I like to travel and work on my own programming
projects.  A Raspberry Pi is running on my closet as a personal
server for backups and matrix home server etc. I am also interested in robotics software in general and automation
technology. 
\section{education}

%%%%%%%%%%%%%%%%%%%%%%%%%%%%%%%%%%%%%%%%%%
%%%%%%%%%%%%%TWENTY LIST ITEMS%%%%%%%%%%%%%%
%%    Four arguments: date; title; where; description %%%%
%%%%%%%%%%%%%%%%%%%%%%%%%%%%%%%%%%%%%%%%%%
\begin{twenty}
 \twentyitem
    {since 2016}
    {M.Sc.}
    {Lund University}
    {Majoring in Physics}
  \twentyitem
    {2013-2016}
    {B.Sc.}
    {Lund University}
    {Majoring in Physics}
  \twentyitem
    {2010-2013}
    {High school}
    {Hagan\"asskolan, \"Almhult}
    {Specializing in natural sciences.}
\end{twenty}


%%%%%%%%%%%%%%%%%%%%%%%%%%%%%%%%%%%%%%%%%%
%%%%%%%%%TWENTY LIST SHORTITEMS%%%%%%%%%%%%%%
%%% Two arguments: date; title/description %%%%%%%%%%
%%%%%%%%%%%%%%%%%%%%%%%%%%%%%%%%%%%%%%%%%%
\section{publications}

\begin{twentyshort}
  \twentyitemshort
  {2016}
  {\href{http://lup.lub.lu.se/student-papers/record/8878322}{Numerical simulations of contact geometry
    effects on transport properties of semiconductor nanowires:}
  Bachelor thesis work at the department of mathematical physics.}
\end{twentyshort}


\section{awards}

\begin{twentyshort}
  \twentyitemshort
    {2013}
    {\href{http://www.siwi.org/latest/vinner-2013-ars-svenska-confidantes/}{Swedish
      Junior Water Prize:} Award for my high school graduation project
    investigating presence of lead in garden hoses.}
\end{twentyshort}


\section{experience}

\begin{twenty}
  \twentyitem
    {2015}
    {Sudoku solver}
    {ios-application}
    {A side project besides my studies to learn about UI programming
      and deployment to ios-devices. (No longer maintained)}
\twentyitem
    {2017}
    {Silly Walks}
    {Artificial neural network}
    {A side project besides my studies to learn about code
      re-usability, software design, artificial intelligence and
      the c++ language. Initially started as part of the Sudoku Solver
    app to import a Sudoku with the camera. (Work in progress)}
\end{twenty}

\section{other information}
Born in the small town of \"Almhult in Sm\aa land, moved to study
physics at Lund University in 2013.

The goal of my studies at Lund University is to train my ability to
learn independently, not to memorize facts and formulas. In an ever
changing world, the ability to quickly comprehend and analyze is more
important than ever. My grades are not spectacular as I never really
study specifically to pass exams, but to understand the
subject. However, I have not to this day failed an exam. 

In ten years I see myself as the specialist, the guy to ask when there
is a problem in my area of expertise. I enjoy investigating abstract
constructions in depth using mathematics to understand them. Although I
am not sure I want to be that person my entire career. Eventually I
want to develop the expertise to maybe start my own enterprise or take
more of a leadership role.

%%%%%%%%%%%%%%%%%%%%
%%%%%ENDBODY%%%%%%%%
%%%%%%%%%%%%%%%%%%%%

\end{document} 
